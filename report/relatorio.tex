\documentclass[12pt]{article}
\usepackage[english]{babel}
\usepackage[utf8]{inputenc}
\usepackage[T1]{fontenc}
\usepackage{geometry}
\usepackage{subcaption}
\usepackage{caption}
\usepackage{float}
\usepackage{tabularx}
\usepackage{graphicx}
\usepackage{multirow}
\usepackage{indentfirst}

\begin{document}

%%fakesection Capa
{
	%%%%%%%%%%%%%%%%%%%%%%%%%%%%%%%%%%%%%%%%%
% Classic Lined Title Page 
% LaTeX Template
% Version 1.0 (27/12/12)
%
% This template has been downloaded from:
% http://www.LaTeXTemplates.com
%
% Original author:
% Peter Wilson (herries.press@earthlink.net)
%
% License:
% CC BY-NC-SA 3.0 (http://creativecommons.org/licenses/by-nc-sa/3.0/)
% 
% Instructions for using this template:
% This title page compiles as is. If you wish to include this title page in 
% another document, you will need to copy everything before 
% \begin{document} into the preamble of your document. The title page is
% then included using \titleAT within your document.
%
%%%%%%%%%%%%%%%%%%%%%%%%%%%%%%%%%%%%%%%%%

%----------------------------------------------------------------------------------------
%	PACKAGES AND OTHER DOCUMENT CONFIGURATIONS
%----------------------------------------------------------------------------------------


%\newcommand*{\plogo}{\fbox{$\mathcal{PL}$}} % Generic publisher logo

%----------------------------------------------------------------------------------------
%	TITLE PAGE
%----------------------------------------------------------------------------------------

\newcommand*{\titleAT}{\begingroup % Create the command for including the title page in the document
\newlength{\drop} % Command for generating a specific amount of whitespace
\drop=0.1\textheight % Define the command as 10% of the total text height

%\rule{\textwidth}{1pt}\par % Thick horizontal line
%\vspace{2pt}\vspace{-\baselineskip} % Whitespace between lines
%\rule{\textwidth}{0.4pt}\par % Thin horizontal line

\centering % Center all text

\vspace{\drop} % Whitespace between the top lines and title
\rule{0.8\textwidth}{0.4pt}\par % Short horizontal line under the title
\vspace{0.25\drop} % Whitespace between the title and short horizontal line

{\Large Projeto de Sistemas Digitais}\\[0.5\baselineskip] % Title line 1
{\Large Trabalho Laboratorial 3}\\[0.75\baselineskip] % Title line 2
\vspace{0.25\drop}
{\Huge DSP IP core for computation of real-time FIR filter}\\

\vspace{0.25\drop} % Whitespace between the title and short horizontal line
\rule{0.8\textwidth}{0.4pt}\par % Short horizontal line under the title
\vspace{\drop} % Whitespace between the thin horizontal line and the author name

{\Large Rui Graça}\par % Author name
{\Large Vinícius Ginja}\par % Author name

\vfill % Whitespace between the author name and publisher text
\includegraphics[scale=1]{./FEUP_Logo.png}\\
%{\large Faculdade de Engenharia\\Universidade do Porto}\par % Publisher

\vspace*{\drop} % Whitespace under the publisher text

%\rule{\textwidth}{0.4pt}\par % Thin horizontal line
\vspace{2pt}\vspace{-\baselineskip} % Whitespace between lines
%\rule{\textwidth}{1pt}\par % Thick horizontal line

\endgroup}

	\pagestyle{empty}
	\titleAT
	\newpage
}
\setcounter{page}{1}

\section{Introduction}

For this project, we set out to optimize a given digital audio IP able to compute digital FIR
filters with up to 16384 coefficients, so that it meets timing requirements established by a 48 KHz
input signal bandwidth. Both input and output are stereo with 18 bit per channel. To achieve this
speed requirement, parallel and pipeline architectures were explored during the development. We also
found different solutions that are fast enough but with unequal FPGA resource utilization. The
results were verified with behavioral and post-synthesis, as well as FPGA implementation using a
signal generator and an oscilloscope. Post-layout simulation failed, probably due to some error in
the model of a RAM block, but tests done on the FPGA show that the behavior is correct.
The input of IP, as well as the filter coefficients, are represented in two's complement with fixed
point with only one non-fractional bit. 

\section{Design decisions}
Since we must be able to compute the response of a filter with up to 16384 coefficients in
$\frac{1}{48}$ ns = 20.8333 $\mu$s, if we set the clock frequency to 100 MHz (clock period of 0.01
$\mu$s), we must have $\frac{16384\cdot0.01}{20.8333} = 7.8$ MAC (Multiply and accumulate) blocks
operating in parallel, computing one product and one accumulation per clock cycle. We conclude
easily that it is not feasible to use the RAM blocks given, since they have a bandwidth of 4
coefficients/data samples per clock cycle, and, to use only 4 MAC blocks in parallel, we would have
to use a clock frequency of 200 MHz in order to achieve the desired timing requirements, which
exceeds the maximum clock frequency in the RAM specification (147 MHz). Our choice was to extend the
RAM bandwidth to 8 coefficients/data samples and to use 8 MAC blocks in parallel. The MAC block used
was the one given, with small modifications, since some of the bits in the accumulator output of the
original block were not used. The MAC block, that is supposed to multiply an input of 18 bits with
one other of 36 bits and then to add the product to an accumulator. The block used divides input of
36 bits in an upper part of 18 bits and in a lower part and executes the multiplication and the
accumulation of both parts in parallel. Because the FGPA used has Hardware optimized for the
computation of MACs with signed 18 bits input (DSP48 blocks), and the lower 18 bits of the 36 bit
input should be used in the MAC in an unsigned fashion, operations with the lower 18 bits were
further divided, being that the lower 17 bits were multiplied with the 18 bits input, being forced
to signed by introducing 0 in the MSB. The 18th bit of the 36 bit input was dealt with individually,
being that its influence in the result is given by a multiplexor that outputs 16b'0 if it 0, or the
16 bit input if it is 1. This allows to take advantage of the usage of the DSP48 blocks, optimizing
the performance of the MAC. This MAC block has 4 pipeline stages Since the inputs of the multiplier
stage are signals of 18 and 36 bits, the output has 54 bits. The output of the MAC is the result of
16384 sums of 54 bit signals, which corresponds to a product of a 54 bit signal with a
$log_2$(16384) = 14 bit signal, and so, it is a signal of 68 bits. However, because this
implementations is supposed to support FIR filters without gain, and the output of the IP is a
signal of 18 bits, the 15 most significant bits of the MAC output are truncated (as well as the
lower 35 bits). Because of this, it is unnecessary to implement signals of 68 bits, because the
occurrence of overflow in the intermediate computations will not affect the final result (unless
overflow occurs in the output, due to filter coefficients that do not respect the no-gain
restriction), and the internal signals of the MAC block were truncated to 53 bits. However, this
change does not result in any performance improvement, since the synthesis tools successfully detect
these bits as unnecessary. Other optimizations were tried in this module, aiming to make the
operation the most suitable possible for the DSP48 blocks, with the least possible logic besides the
DSP48 itself. However, we soon realized that the block is already quite optimized, and that the RAM
blocks will introduce the bottleneck to the critical path, so that optimization in the MAC speed
would not be a major benefit for the project.

As referred, both the circular buffer RAM and the RAM for the coefficients were extended to support
a bandwidth of 8 coefficients/data samples per clock cycle. This was done by keeping the structure
of the original blocks, but adjusting the parameters in order to change the bandwidth. In the
circular buffer, it was not required to keep the two physical blocks for each logic RAM block
required in the original circular buffer, causing the resulting design to be somewhat simplified in
relation to the original.

The final IP is constituted by 8 MAC blocks operating in parallel, followed by the sum of the output
of all the MAC blocks. We have tested three options for this sum: using only one pipeline stage, in
which all the 8 signals from the MAC outputs are summed, using two pipeline stages, being that the
MAC outputs are summed two by two in one stage, and the 4 signals that result from that stage are
summed in the following stage, and using three pipeline stages, being that the output of the MAC
blocks are summed two by two in the first stage, the 4 outputs of this stage are summed two by two
in the second stage, and the two signals that result are summed in the last stage. These three
options were compared relatively to time performance and resource usage. Moreover, we have
implemented control mechanism for the reset of the MAC blocks in the end of the computation of one
output sample and a shift register to set the output ready signal in the clock cycle that the right
result outputs the last pipeline stage.

\section{Simulation and Results}
Simulations were done using the testbench given, comparing the output of the filter with a golden
output, for a given set of coefficients. We have confirmed that the output equals the golden output
all the values tested in the testbench. In the first design stages of the project, simulations were
performed only at a behavioral level, whereas in later stages, post-synthesis and post-translate
simulations were done. Although these resulted in the expected correct response, post-routing
simulations resulted in undefined results at the output, due to some problem in the RAM. We believe
that this is due to some problem in the RAM cell module, and not to the project itself, since a
correct behavior was obtained in an FPGA.

As referred, several design options were compared. Using three pipeline stages for the summation of
the MAC results, we obtained the following results, by post synthesis analysis, with optimization
goal set as time, with high effort:

\begin{verbatim}
Maximum Clock Frequency: 125.455 MHz (7.971ns)

Logic Utilization                   Used    Available   Utilization
Number of Slice Registers           3711    54576       6%
Number of Slice LUTs                5306    27288       19%
Number of fully used LUT-FF pairs   3631    5386        67%
Number of Block RAM/FIFO            96      116         82%
Number of BUFG/BUFGCTRLs            1       16          6%
Number of DSP48A1s                  32      58          55%
\end{verbatim}

With the same configuration, but with area as the optimization goal, we obtained:

\begin{verbatim}
Maximum Clock Frequency:  118.99 (8.404 ns)

Logic Utilization                   Used    Available   Utilization
Number of Slice Registers           3646    54576       6%
Number of Slice LUTs                5297    27288       19%
Number of fully used LUT-FF pairs   3616    5327        67%
Number of Block RAM/FIFO            96      116	        82%
Number of BUFG/BUFGCTRLs            1       16          6%
Number of DSP48A1s                  32      58          55%
\end{verbatim}

Using 2 pipeline stages in the MAC, with speed as the optimization goal, we obtained:
	
\begin{verbatim}
Maximum Clock Frequency: 121.908MHz (8.203ns)

Logic Utilization                   Used    Available   Utilization
Number of Slice Registers           3490    54576       6%
Number of Slice LUTs                5303    27288       19%
Number of fully used LUT-FF pairs   3266    5527        59%
Number of Block RAM/FIFO            96      116         82%
Number of BUFG/BUFGCTRLs            1       16          6%
Number of DSP48A1s                  32      58          55%
\end{verbatim}

And with area as the optimization goal:

\begin{verbatim}
Maximum Clock Frequency: 118.948 MHz (8.407ns)

Logic Utilization                   Used    Available   Utilization
Number of Slice Registers           3433    54576       6%
Number of Slice LUTs                5297    27288       19%
Number of fully used LUT-FF pairs   3404    5326        63%	
Number of Block RAM/FIFO            96      116         82%
Number of BUFG/BUFGCTRLs            1       16          6%
Number of DSP48A1s                  32      58          55%
\end{verbatim}

Using only one pipeline stage, with speed as optimization goal, we obtained:

\begin{verbatim}
Maximum Clock Frequency: 107.980 MHz (9.261ns)

Logic Utilization                   Used    Available   Utilization
Number of Slice Registers           3049    54576       5%
Number of Slice LUTs                5294    27288       19%
Number of fully used LUT-FF pairs   2593    5750        45%
Number of Block RAM/FIFO            96      116         82%
Number of BUFG/BUFGCTRLs            1       16          6%
Number of DSP48A1s                  32      58          55%
\end{verbatim}

And using area as the optimization goal:

\begin{verbatim}
Maximum Clock Frequency: 108.036MHz (9.256ns)

Logic Utilization                   Used    Available   Utilization
Number of Slice Registers           3010    54576       5%
Number of Slice LUTs                5297    27288       19%
Number of fully used LUT-FF pairs   2980    5327        55%
Number of Block RAM/FIFO            96      116         82%
Number of BUFG/BUFGCTRLs            1       16          6%
Number of DSP48A1s                  32      58          55%
\end{verbatim}

Based on these results, the usage of 2 stages seemed a reasonable option, and it was used for the
generation of the next stages of the implementation. However, post-routing simulations suggest that
the usage of three stages is a better option, when trying to accomplish the desired time
performance.

Optimizing with a goal of 100 MHz for the clock frequency, we obtained:

\begin{verbatim}
Maximum Clock Frequency: 100.311MHz (9.969ns)

Slice Logic Utilization         Used    Available   Utilization
Number of Slice Registers       3590    54576       6% 
Number of Slice LUTs            4492    27288       16%  
Number of occupied Slices       1409    6822        20%
Number of MUXCYs used           3208    13644       23% 
Number of LUT Flip Flop pairs   5054
Number of RAMB16BWERs           96      116         82%
Number of BUFG/BUFGMUXs         1       16          6%
Number of DSP48A1s              32      58          55%
\end{verbatim}

The tools were, however, unable to optimize to a goal of 111.11 MHz, resulting in:
 
\begin{verbatim}
Maximum Clock Frequency: 104.178MHz (9.599ns)

Slice Logic Utilization         Used    Available   Utilization
Number of Slice Registers       3590    54576       6% 
Number of Slice LUTs            4496    27288       16% 
Number of occupied Slices       1444    6822        21%
Number of MUXCYs used           3208    13644       23%
Number of LUT Flip Flop pairs   4962
Number of RAMB16BWERs           96      116         82%
Number of BUFG/BUFGMUXs         1       16          6% 
Number of DSP48A1s              32      58          55%
\end{verbatim}

Using three pipeline stages, this goal was successfully achieved, with the following results:

\begin{verbatim}
Maximum Clock Frequency: 111.520MHz (8.967ns)

Slice Logic Utilization         Used    Available   Utilization
Number of Slice Registers       3712    54576       6%
Number of Slice LUTs            4612    27288       16%
Number of occupied Slices       1466    6822        21%
Number of MUXCYs used3320       13644   24%
Number of LUT Flip Flop pairs   5060
Number of RAMB16BWERs           96      116         82% 
Number of BUFG/BUFGMUXs         1       16          6% 
Number of DSP48A1s              32      58          55%
\end{verbatim}

Optimization for higher frequencies could not be successfully obtained by the tools.

We see that with a small increase in the resource utilization, a higher frequency was obtained by
the design with three pipeline stages. This implementation was tested in the FPGA, using a signal
generator to generate the input and an oscilloscope to measure the output. We observed that the
filter behavior was the expected when introducing both sine and square waves. However, we observed
that overflow occurred when signals over a certain threshold were used. Analyzing the filter
response, we observed that it shows gain for certain frequencies, which explains the overflow in the
output.

 
\end{document}
